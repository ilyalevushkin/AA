\documentclass[12pt]{report}

\usepackage{cmap}
\usepackage[T2A]{fontenc}
\usepackage[utf8]{inputenc}
\usepackage[english,russian]{babel}

\usepackage{amssymb}

\usepackage{amsmath}
\usepackage{array}

\usepackage{csquotes}


\usepackage{longtable}

\usepackage{graphicx}
\usepackage{float}%"Плавающие" картинки
\usepackage{wrapfig}%Обтекание фигур (таблиц, картинок и прочего)
\graphicspath{{images/}}
\usepackage[usenames]{color}

\usepackage{listings}

\lstset{
language=c++,                 % выбор языка для подсветки (здесь это c++)
basicstyle=\small\sffamily, % размер и начертание шрифта для подсветки кода
numbers=left,               % где поставить нумерацию строк (слева\справа)
numberstyle=\tiny,           % размер шрифта для номеров строк
stepnumber=1,                   % размер шага между двумя номерами строк
numbersep=5pt,                % как далеко отстоят номера строк от подсвечиваемого кода
showspaces=false,            % показывать или нет пробелы специальными отступами
showstringspaces=false,      % показывать или нет пробелы в строках
showtabs=false,             % показывать или нет табуляцию в строках
frame=single,              % рисовать рамку вокруг кода
tabsize=1,                 % размер табуляции по умолчанию равен 2 пробелам
captionpos=t,              % позиция заголовка вверху [t] или внизу [b] 
breaklines=true,           % автоматически переносить строки (да\нет)
breakatwhitespace=false, % переносить строки только если есть пробел
escapeinside={\#*}{*)}   % если нужно добавить комментарии в коде
}


% Для измененных титулов глав:
\usepackage{titlesec, blindtext, color} % подключаем нужные пакеты
\definecolor{gray75}{gray}{0.75} % определяем цвет
\newcommand{\hsp}{\hspace{20pt}} % длина линии в 20pt
% titleformat определяет стиль
\titleformat{\chapter}[hang]{\Huge\bfseries}{\thechapter\hsp\textcolor{gray75}{|}\hsp}{0pt}{\Huge\bfseries}


\usepackage{pgfplots}
\usepackage{filecontents}
\usetikzlibrary{datavisualization}
\usetikzlibrary{datavisualization.formats.functions}



\begin{filecontents}{Shell_best.dat}
	100 13577
	200 34494
	300 60990
	400 80575
	500 99125
	600 133820
	700 154214
	800 173332
	900 197332
	1000 157747
\end{filecontents}

\begin{filecontents}{Shell_random.dat}
	100 59134
	200 139994
	300 215931
	400 311772
	500 361400
	600 445444
	700 514780
	800 642633
	900 765237
	1000 779791
\end{filecontents}

\begin{filecontents}{Shell_worst.dat}
	100 32601
	200 76995
	300 129692
	400 164619
	500 230048
	600 265233
	700 334282
	800 345377
	900 448265
	1000 457217
\end{filecontents}

\begin{filecontents}{Gnom_best.dat}
	100 2156
	200 4333
	300 6462
	400 7954
	500 10781
	600 12787
	700 15001
	800 17273
	900 19385
	1000 21410
\end{filecontents}


\begin{filecontents}{Gnom_random.dat}
	100 207040
	200 668476
	300 1429117
	400 2280314
	500 3699399
	600 5200664
	700 6985759
	800 9273739
	900 11490163
	1000 13970190
\end{filecontents}


\begin{filecontents}{Gnom_worst.dat}
	100 386612
	200 1197562
	300 2632503
	400 4648589
	500 7129986
	600 10339203
	700 13717497
	800 17604188
	900 23067462
	1000 27748485
\end{filecontents}


\begin{filecontents}{Hairbrush_best.dat}
	100 32989
	200 78287
	300 125411
	400 174387
	500 230146
	600 274164
	700 312998
	800 355905
	900 431985
	1000 438616
\end{filecontents}


\begin{filecontents}{Hairbrush_random.dat}
	100 55036
	200 128151
	300 204941
	400 289040
	500 345098
	600 423519
	700 484999
	800 553373
	900 639390
	1000 746300
\end{filecontents}


\begin{filecontents}{Hairbrush_worst.dat}
	100 27574
	200 93381
	300 150004
	400 205934
	500 266063
	600 333236
	700 355299
	800 426127
	900 458591
	1000 512719
\end{filecontents}






\begin{document}
	\begin{titlepage}
		\centering
		{\scshape\LARGE МГТУ им. Баумана \par}
		\vspace{3cm}
		{\scshape\Large Лабораторная работа №3\par}
		\vspace{0.5cm}	
		{\scshape\Large По курсу: "Анализ алгоритмов"\par}
		\vspace{1.5cm}
		{\huge\bfseries Алгоритмы сортировок\par}
		\vspace{2cm}
		{\Large Работу выполнил: Левушкин Илья, ИУ7-52Б\par}
		\vspace{0.5cm}
		{\Large Преподаватели:  Волкова Л.Л., Строганов Ю.В.\par}
		
		\vfill
		\large \textit {Москва, 2019} \par
	\end{titlepage}
	
	\tableofcontents
	
	\newpage
	\chapter*{Введение}
	
	~\
	
	\addcontentsline{toc}{chapter}{Введение}
	Алгоритмы сортировки находят широкое применение во многих сферах,
	поэтому сортировки являются одной из наиболее обширных и проработанных областей информатики.
	
	В данной работе требуется реализовать и оценить трудоемкость трех
	алгоритмов сортировок:
	
	\begin{itemize}
		\item сортировка Шелла;
		\item гномья сортировка;
		\item сортировка расческой.
	\end{itemize}

   {\bf Цель работы: изучение алгоритмов сортировки.}
   
   {\bf Задачи работы:}
   
   \begin{itemize}
   	\item разработка и реализация алгоритмов;
   	\item исследование временных затрат алгоритмов;
   	\item описание и обоснование полученных результатов.
   \end{itemize}

	\chapter{Аналитический раздел}
	
	~\
	
	В данном разделе будут описаны алгоритмы сортировки Шелла, гномьей сортировки и сортировки расческой.
	
	\section{Описание алгоритмов}
	
	~\
	
	Задача сортировки формулируется следующим образом: дана последовательность элементов
	\begin{equation}\label{eq:1}
		a_{1}, a_{2},\cdots, a_{n}
	\end{equation}
	
	Требуется упорядочить элементы по не убыванию или по не возрастанию - найти перестановку $(i_{1}, i_{2}, \cdots, i_{n})$ ключей \ref{eq:1}, либо по неубыванию:
	
	\begin{equation}\label{eq:2}
	a(i_{1}) \leq a(i_{2}) \leq \cdots \leq a(i_{n})
	\end{equation}
	
	либо по не возрастанию:
	
	\begin{equation}\label{eq:3}
	a(i_{1}) \geq a(i_{2}) \geq \cdots \geq a(i_{n})
	\end{equation}
	
	\section{Сортировка Шелла}
	
	~\
	
	Сортировка Шелла — это алгоритм сортировки, являющийся усовершенствованным вариантом сортировки вставками. Идея метода Шелла состоит в сравнении элементов, стоящих не только рядом, но и на определённом расстоянии друг от друга. При сортировке Шелла сначала сравниваются и сортируются между собой значения, стоящие один от другого на
	некотором расстоянии d. После этого процедура повторяется для некоторых
	меньших значений d, а завершается сортировка Шелла упорядочивающем
	элементов при d = 1, то есть обычной сортировкой вставками.
	
	Среднее время работы алгоритма зависит от длин промежутков — d, на
	которых будут находиться сортируемые элементы исходного массива емкостью N на каждом шаге алгоритма. Первоначально используемая Шеллом
	последовательность длин промежутков:
	
	\begin{equation} \label{eq:4}
	d_{1} = \frac{N}{2}, d_{i} = \frac{d_{i-1}}{2}, d_{k} = 1
	\end{equation}
	
	\section{Гномья сортировка}
	
	~\
	
	Гномья сортировка — это алгоритм сортировки, похожий на сортировку вставками, но в отличие от последней перед вставкой на нужное место
	происходит серия обменов, как в сортировке пузырьком. Алгоритм находит первое место, где два соседних элемента стоят в неправильном порядке
	и меняет их местами. Он пользуется тем фактом, что обмен может породить новую пару, стоящую в неправильном порядке, только до или после
	переставленных элементов. Он не допускает, что элементы после текущей
	позиции отсортированы, таким образом, нужно только проверить позицию до переставленных элементов.
	
	\section{Сортировка расческой}
	
	~\
	
	Сортировка расческой является улучшенной модификацией сортировки пузырьком. В сортировке пузырьком, когда сравниваются два элемента,
	промежуток между элементами равен единице. Основная идея алгоритма
	сортировки расческой заключается в том, чтобы первоначально брать достаточно большое расстояние между сравниваемыми элементами и по мере
	упорядочивания массива сужать это расстояние вплоть до минимального.
	Первоначальный разрыв между сравниваемыми элементами лучше брать
	с учётом специальной величины, называемой фактором уменьшения, оптимальное значение которой равно примерно:
	
	\begin{gather}\label{eq:5}
	\frac{1}{1-e^\phi} = 1,247 \cdots,
	\intertext{Где}
	\begin{tabular}{>{$}r<{$}@{\ :\ }l}
	e & основание натурального логарифма\\
	\phi & золотое сечение
	\end{tabular}\nonumber
	\end{gather}
	
	Начальное расстояние между элементами равно размеру массива, разделённого на фактор уменьшения. Массив проходится с этим шагом, после
	чего шаг делится на фактор уменьшения и проход по списку повторяется вновь. Так продолжается до тех пор, пока разность индексов не достигает единицы. После этого массив продолжает упорядочивание обычным
	пузырьком.
	
	\section{Выводы по аналитической части}
	
	~\
	
	В данном разделе были описаны алгоритмы сортировки Шелла, быстрой сортировки и сортировки расческой.
	
	\chapter{Конструкторский раздел}
	
	~\
	
	В данном разделе в соответствии с описанием алгоритмов, приведенными в аналитической части работы, будут рассмотрены схемы алгоритмов
	сортировки двух массивов данных, а также будет произведен расчёт сложности алгоритмов.
	
	\section{Схемы алгоритмов}
	
	~\
	
	В данном пункте представлены схемы алгоритмов сортировки Шелла (\ref{ris:image1}), гномьей сортировки (\ref{ris:image2}) и сортировки расческой (\ref{ris:image3})
	
	
	\begin{figure}[H]
		\centering
		\includegraphics[width=0.22\linewidth]{Shell}
		\caption{Алгоритм сортировки Шелла.}
		\label{ris:image1}
	\end{figure}

\newpage

	\begin{figure}[H]
		\centering
		\includegraphics[width=0.7\linewidth]{Gnome}
		\caption{Алгоритм гномьей сортировки.}
		\label{ris:image2}
	\end{figure}

\newpage

	 \begin{figure}[H]
		\centering
		\begin{minipage}[h!]{0.59\linewidth}
			\includegraphics[width=0.75\linewidth]{Hairbrush}
		\end{minipage}
		\begin{minipage}[h!]{0.39\linewidth}
			\includegraphics[width=1.5\linewidth]{Hairbrush2}
		\end{minipage}
		\caption{Алгоритм сортировки расческой.}
		\label{ris:image3}
	\end{figure}

	
	
	\section{Расчет сложности алгоритмов}
	
	\subsection{Модель вычислений}
	
	~\
	
	Для корректного расчёта сложности алгоритмов следует описать модель
	вычислений. Пусть любые арифметические операции имеют стоимость 1.
	Стоимость условного перехода if-else возьмем за 0 и будем учитывать лишь
	стоимость вычисления логического выражения. Цикл начинается с инициализации и проверки, стоимость которых в сумме составляет 2. В каждой
	итерации цикла происходит проверка условия и увеличение счётчика, каждый из которых имеет стоимость 1, соответственно, каждая итерация цикла
	имеет добавочную стоимость 2.
	
	\subsection{Сортировка Шелла ~\cite{first}}
	\begin{itemize}
		\item Лучший случай $\sim \Theta(n)$
		\item Худший случай $\sim \Theta(n\log^2n)$
	\end{itemize}

	\subsection{Гномья сортировка}
	\begin{itemize}
		\item Лучший случай - $ 2 + (n - 1) * 3 = 3n - 1 \sim \Theta(n)$
		\item Худший случай - $2 + (n - 1)(n - 1)(3 + 10 + 1 + 1) = 15n^2 - 30n + 17 \sim \Theta(n^2)$
	\end{itemize}

	\subsection{Сортировка расческой ~\cite{first}}
	\begin{itemize}
		\item Лучший случай $\sim \Theta(n\log n)$
		\item Худший случай $\sim \Theta(n^2)$
	\end{itemize}

	\section{Вывод из конструкторской части}
	
	~\
	
	В данном разделе были рассмотрены схемы алгоритмов сортировки Шелла, гномьей сортировки и сортировки расческой, а также был произведен
	расчёт сложности алгоритмов.
	
	
	\chapter{Технологический раздел}
	
	~\
	
	В данном разделе будут рассмотрены требования к разрабатываемому
	программному обеспечению, средства, использованные в процессе разработки для реализации поставленных задач, а также представлен листинг кода
	программы.
	
	\section{Требования к программному обеспечению}
	
	~\
	
	Программное обеспечение должно реализовывать 3 алгоритма сортировки - сортировка Шелла, гномья сортировка и сортировка расческой. Пользователь должен иметь возможность произвести вычисления для массивов,
	размер которых он вводит, а также иметь возможность сравнить время работы этих алгоритмов.
	
	\section{Средства реализации}
	
	~\
	
	Для реализации поставленной задачи был использован язык программирования C++ ~\cite{second}. Проет был выполнен в среде QT Creator ~\cite{third}. Для измерения процессорного времени была использована ассемблерная инструкция
	rdtsc ~\cite{fourth}.
	
	\section{Листинг кода}
	
	На основе схем алгоритмов, представленных в конструкторском разделе, в соответствии с указанными требованиями к реализации с использованием средств языка C++ было разработано программное обеспечение,
	содержащее реализации выбранных алгоритмов. В данном пункте приведён листинг этих реализаций: листинги \ref{some-code1}, \ref{some-code2}, \ref{some-code3}.
	
	\begin{lstlisting}[label=some-code1,caption=Алгоритм сортировки Шелла]
	void ShellSort (vector<int>& array)
	{
	int tmp;
	int size = array.size();
	for (int step = size / 2; step > 0; step /= 2)
	{
	for (int i = step; i < size; i++)
	{
	for (int j = i - step; j >= 0 && array[size_t(j)] > array[size_t(j + step)]; j -= step)
	{
	tmp = array[size_t(j)];
	array[size_t(j)] = array[size_t(j + step)];
	array[size_t(j + step)] = tmp;
	}
	}
	}
	}
	\end{lstlisting}
	
	\begin{lstlisting}[label=some-code2,caption=Алгоритм сортировки расческой]
	void comb(vector<int>& sort)
	{
	int size = sort.size();
	double fakt = 1.2473309;
	int step = size - 1;
	
	while (step >= 1)
	{
	step /= fakt;
	for (int i = 0; i < size - step; i++)
	{
	if (sort[i] > sort[i + step])
	{
	swap(sort[i], sort[i + step]);
	}
	}
	}

	for (int i = 0; i < size - 1; i++)
	{
	bool swapped = false;
	for (int j = 0; j < size - i - 1; j++)
	{
	if (sort[j] > sort[j + 1])
	{
	swap(sort[j], sort[j + 1]);
	swapped = true;
	}
	}
	if (!swapped)
	break;
	}
	}
	\end{lstlisting}
	
	\begin{lstlisting}[label=some-code3,caption=Алгоритмгномьей сортировки]
	void gnome_sort(vector<int>& sort)
	{
	int size = sort.size();
	for (size_t i = 0; i < size - 1; i++)
	{
	if (sort[i] > sort[i + 1])
	{
	swap(sort[i], sort[i + 1]);
	if (i != 0)
	{
	i -= 2; 
	}
	}
	}
	}
	\end{lstlisting}
	
	\section{Выводы по технологической части}
	
	~\
	
	В данном разделе были рассмотрены требования к разрабатываемому
	программному обеспечению, средства, использованные в процессе разработки, а также был представлен листинг реализаций выбранных алгоритмов.
	
	\chapter{Экспериментальный раздел}
	
	~\
	
	В данном разделе будет проведено исследование временных затрат разработанного программного обеспечения, вместе с подробным сравнительным анализом реализованных алгоритмов на основе экспериментальных
	данных.
	
	\section{Сравнительное исследование}
	
	~\
	
	Замеры времени выполнялись на массивах размера от 100 до 1000 с
	шагом 100 для сортировки Шелла (табл. 4.1.1, рис. 4.1.1), гномьей сортировки (табл. 4.1.2, рис. 4.1.2) и сортировки расческой (табл. 4.1.3, рис.
	4.1.3). Числа в матрицах генерировались случайным образом.
	
	\newpage
	
	\begin{table}[h]
		\caption{\label{tab:time1} Сравнение времени работы алгоритма сортировки Шелла в тактах процессора. }
		\begin{center}
			\begin{tabular}{|c|c|c|p{3.8cm}|}
				\hline
				Размер массива & Лучший сл.& Средний сл. & Худший сл. \\ [0.5ex] 
				\hline
				100 & 13577 & 59134 & 32601\\
				\hline
				200 & 34494 & 139994 & 76995\\
				\hline
				300 & 60990 & 215931 & 129692\\
				\hline
				400 & 80575 & 311772 & 164619\\
				\hline
				500 & 99125 & 361400 & 230048\\
				\hline
				600 & 133820 & 445444 & 265233\\
				\hline
				700 & 154214 & 514780 & 334282\\
				\hline
				800 & 173332 & 642633 & 345377\\
				\hline
				900 & 197332 & 765237 & 448265\\
				\hline
				1000 & 157747 & 779791 & 457217\\
				\hline
			\end{tabular}
		\end{center}
	\end{table}
	
	
	\begin{tikzpicture}
	
	\begin{axis}[
	axis lines = left,
	xlabel = {$size$, чисел},
	ylabel = {$time$, тактов},
	legend pos=north west,
	ymajorgrids=true
	]
	\addplot[color=red] table[x index=0, y index=1] {Shell_best.dat};
	\addplot[color=green, mark=square] table[x index=0, y index=1] {Shell_random.dat};
	\addplot[color=blue, mark=square] table[x index=0, y index=1] {Shell_worst.dat};
	
	\addlegendentry{Best}
	\addlegendentry{Random}
	\addlegendentry{Worst}
	
	\end{axis}
	\end{tikzpicture}
	
	\newpage
	
	\begin{table}[h]
		\caption{\label{tab:time2} Сравнение времени работы алгоритма гномьей сортировки в тактах процессора }
		\begin{center}
			\begin{tabular}{|c|c|c|p{3.8cm}|}
				\hline
				Размер массива & Лучший сл.& Средний сл. & Худший сл. \\ [0.5ex] 
				\hline
				100 & 2156 &207040 & 386612\\
				\hline
				200 & 4333 & 668476 & 1197562\\
				\hline
				300 & 6462 & 1429117 & 2632503\\
				\hline
				400 & 7954 & 2280314 & 4648589\\
				\hline
				500 & 10781 & 3699399 & 7129986\\
				\hline
				600 & 12787 & 5200664 & 10339203\\
				\hline
				700 & 15001 & 6985759 & 13717497\\
				\hline
				800 & 17273 & 9273739 & 17604188\\
				\hline
				900 & 19385 & 11490163 & 23067462\\
				\hline
				1000 & 21410 & 13970190 & 27748485\\
				\hline
			\end{tabular}
		\end{center}
	\end{table}
	
	\newpage
	
	\begin{tikzpicture}
	
	\begin{axis}[
	axis lines = left,
	xlabel = {$size$, чисел},
	ylabel = {$time$, тактов},
	legend pos=north west,
	ymajorgrids=true
	]
	\addplot[color=red] table[x index=0, y index=1] {Gnom_best.dat};
	\addplot[color=green, mark=square] table[x index=0, y index=1] {Gnom_random.dat};
	\addplot[color=blue, mark=square] table[x index=0, y index=1] {Gnom_worst.dat};
	
	\addlegendentry{Best}
	\addlegendentry{Random}
	\addlegendentry{Worst}
	
	\end{axis}
	\end{tikzpicture}
	
	\begin{tikzpicture}
	\begin{axis}[
		axis lines = left,
		xlabel = {$size$, чисел},
		ylabel = {$time$, тактов},
		legend pos=north west,
		ymajorgrids=true
		]
		\addplot[color=red] table[x index=0, y index=1] {Gnom_best.dat};
		
		\addlegendentry{Gnome\_best}
		
	\end{axis}
	\end{tikzpicture}
	
	\newpage
	
	\begin{table}[h]
		\caption{\label{tab:time3} Сравнение времени работы алгоритма сортировки расчески в
			12
			тактах процессора }
		\begin{center}
			\begin{tabular}{|c|c|c|p{3.8cm}|}
				\hline
				Размер массива & Лучший сл.& Средний сл. & Худший сл. \\ [0.5ex] 
				\hline
				100 & 32989 & 55036 & 27574\\
				\hline
				200 & 78287 & 128151 & 93381\\
				\hline
				300 & 125411 & 204941 & 150004\\
				\hline
				400 & 174387 & 289040 & 205934\\
				\hline
				500 & 230146 & 345098 & 266063\\
				\hline
				600 & 274164 & 423519 & 333236\\
				\hline
				700 & 312998 & 484999 & 355299\\
				\hline
				800 & 355905 & 553373 & 426127\\
				\hline
				900 & 431985 & 639390 & 458591\\
				\hline
				1000 & 438616 & 746300 & 512719\\
				\hline
			\end{tabular}
		\end{center}
	\end{table}
	
	\begin{tikzpicture}
	
	\begin{axis}[
	axis lines = left,
	xlabel = {$size$, чисел},
	ylabel = {$time$, тактов},
	legend pos=north west,
	ymajorgrids=true
	]
	\addplot[color=red] table[x index=0, y index=1] {Hairbrush_best.dat};
	\addplot[color=green, mark=square] table[x index=0, y index=1] {Hairbrush_random.dat};
	\addplot[color=blue, mark=square] table[x index=0, y index=1] {Hairbrush_worst.dat};
	
	\addlegendentry{Best}
	\addlegendentry{Random}
	\addlegendentry{Worst}
	
	\end{axis}
	\end{tikzpicture}
	
	
	\section{Выводы по экспериментальному разделу}
	
	~\
	
	В данном разделе было проведено исследование временных затрат разработанного программного обеспечения, вместе с подробным сравнительным анализом реализованных алгоритмов на основе экспериментальных
	данных.
	
	\chapter*{Заключение}
	
	~\
	
	В ходе выполнения данной лабораторной работы были изучены и реализованы алгоритмы сортировки. Алгоритмы были разработаны и реализацованы, было проведено исследование временных затрат алгоритмов, а
	также дано описание и обоснование полученных результатов.
	
	\vspace{0.5cm}
	В аналитическом разделе было дано описание стандартного алгоритма
	и алгоритма Винограда. В конструкторском разделе был формализован и
	описан процесс вычисления пороизвдения двух матриц, разработаны алгоритмы. В технологическом разделе были рассмотрены требования к разрабатываемому программному обеспечению, средства, использованные в процессе разработки для реализации поставленных задач, а также представлен
	листинг кода программы. В экспериментальном разеде было проведено исследование временных затрат.
	
	\begin{thebibliography}{}
		\bibitem{first} Дж. Макконелл. Анализ алгоритмов. Вводный курс. — Москва: Техносфера, 2004. — ISBN 5-94836-005-9.
		\bibitem{second} ISO/IEC JTC1 SC22 WG21 N 3690 «Programming Languages — C++» [Электронный ресурс]. – Режим доступа: https://devdocs.io/cpp/, свободный. (Дата обращения: 29.09.2019 г.)
		\bibitem{third} QT Creator Manual [Электронный ресурс]. – Режим
		доступа: https://doc.qt.io/qtcreator/index.html, свободный. (Дата
		обращения: 29.09.2019 г.)
		\bibitem{fourth} Microsoft «rdtsc» [Электронный ресурс]. – Режим доступа:
		https://docs.microsoft.com/ru-ru/cpp/intrinsics/rdtsc?view=vs-2019,
		свободный. (Дата обращения: 29.09.2019 г.)
	\end{thebibliography}
	\addcontentsline{toc}{chapter}{Литература}

\end{document}